\documentclass[11pt]{article}

% Language setting
% Replace `english' with e.g. `spanish' to change the document language
\usepackage[english]{babel}

% Set page size and margins
% Replace `letterpaper' with `a4paper' for UK/EU standard size
\usepackage[letterpaper,top=3cm,bottom=3cm,left=3cm,right=3cm,marginparwidth=1.75cm]{geometry}

% Useful packages
\usepackage{amsmath}
\usepackage{graphicx}
\usepackage[colorlinks=true, allcolors=blue]{hyperref}

\begin{document}

\begin{titlepage}
      \begin{center}
        {\Large \textbf{Computing with real irreducible representations of finite groups}}
    
        \vspace{3cm}
    
        \includegraphics[width=64mm]{oxlogo.png}
    
        \vspace{3cm}
    
        %{\large
        %Pavol Kollár \par
        %Corpus Christi College \par
        %University of Oxford \par
        %\vspace{1cm}
        %Supervisor: Dr. Dmitrii Pasechnik
        %}
        {\large Candidate Number: XXXXXXX}
    
        \vspace{3cm}
    
        \emph{Final Honour School of Mathematics and Computer Science Part C} \par
        Trinity 2023 \par
    
      \end{center}
\end{titlepage}

\newpage

\section{Abstract}
Representation theory of finite groups over complex numbers is very well understood, in particular
it is classically known that all such representations of a group $G$ of exponent $n$ may be
realised over cyclotomic numbers of degree $n$. Real representations, i.e. these ones realised
by matrices with real entries, while important in applications, are understood much less.

It was open for a long time whether real representations of $G$ may be realised
over degree $n$ real cyclotomics; it was recently shown by Dmitrii Pasechnik to be
the case, cf. \cite{Pas21} \url{https://www.ams.org/journals/ert/2021-25-31/S1088-4165-2021-00587-X/}

The latter describes an algorithm to produce such a realisation. The main part of the
project will be to produce an implementation of this algorithm in one of computer
algebra system, most probably in GAP or in Sagemath.

The project can be viewed as a continuation of work carried out by Dmitrii Pasechnik
and his student Kaashif Hymabaccus in \url{https://joss.theoj.org/papers/10.21105/joss.01835}

\textbf{Prerequisites}: some maths background, in particular in algebra, interest in open-source software.

\newpage

\tableofcontents

\newpage

\section{Introduction}

Welcome.

\section{Background}

\subsection{Representation Theory}

Some information about representation theory.

\subsection{GAP}

Some information about GAP.

\section{Computing real representations}

\subsection{Dmitrii's algorithm}

First algorithm, as outlined by the paper.

\subsection{Further speed improvements}

E.g. Kronecker trick on $P$ directly.

\section{Further comments}

In case there are some other bits that need mentioning. Possibly some code-appendix.

\bibliographystyle{unsrt}
\bibliography{sample}

\end{document}